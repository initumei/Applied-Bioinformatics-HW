
% Default to the notebook output style

    


% Inherit from the specified cell style.




    
\documentclass[11pt]{article}

    
    
    \usepackage[T1]{fontenc}
    % Nicer default font (+ math font) than Computer Modern for most use cases
    \usepackage{mathpazo}

    % Basic figure setup, for now with no caption control since it's done
    % automatically by Pandoc (which extracts ![](path) syntax from Markdown).
    \usepackage{graphicx}
    % We will generate all images so they have a width \maxwidth. This means
    % that they will get their normal width if they fit onto the page, but
    % are scaled down if they would overflow the margins.
    \makeatletter
    \def\maxwidth{\ifdim\Gin@nat@width>\linewidth\linewidth
    \else\Gin@nat@width\fi}
    \makeatother
    \let\Oldincludegraphics\includegraphics
    % Set max figure width to be 80% of text width, for now hardcoded.
    \renewcommand{\includegraphics}[1]{\Oldincludegraphics[width=.8\maxwidth]{#1}}
    % Ensure that by default, figures have no caption (until we provide a
    % proper Figure object with a Caption API and a way to capture that
    % in the conversion process - todo).
    \usepackage{caption}
    \DeclareCaptionLabelFormat{nolabel}{}
    \captionsetup{labelformat=nolabel}

    \usepackage{adjustbox} % Used to constrain images to a maximum size 
    \usepackage{xcolor} % Allow colors to be defined
    \usepackage{enumerate} % Needed for markdown enumerations to work
    \usepackage{geometry} % Used to adjust the document margins
    \usepackage{amsmath} % Equations
    \usepackage{amssymb} % Equations
    \usepackage{textcomp} % defines textquotesingle
    % Hack from http://tex.stackexchange.com/a/47451/13684:
    \AtBeginDocument{%
        \def\PYZsq{\textquotesingle}% Upright quotes in Pygmentized code
    }
    \usepackage{upquote} % Upright quotes for verbatim code
    \usepackage{eurosym} % defines \euro
    \usepackage[mathletters]{ucs} % Extended unicode (utf-8) support
    \usepackage[utf8x]{inputenc} % Allow utf-8 characters in the tex document
    \usepackage{fancyvrb} % verbatim replacement that allows latex
    \usepackage{grffile} % extends the file name processing of package graphics 
                         % to support a larger range 
    % The hyperref package gives us a pdf with properly built
    % internal navigation ('pdf bookmarks' for the table of contents,
    % internal cross-reference links, web links for URLs, etc.)
    \usepackage{hyperref}
    \usepackage{longtable} % longtable support required by pandoc >1.10
    \usepackage{booktabs}  % table support for pandoc > 1.12.2
    \usepackage[inline]{enumitem} % IRkernel/repr support (it uses the enumerate* environment)
    \usepackage[normalem]{ulem} % ulem is needed to support strikethroughs (\sout)
                                % normalem makes italics be italics, not underlines
    

    
    
    % Colors for the hyperref package
    \definecolor{urlcolor}{rgb}{0,.145,.698}
    \definecolor{linkcolor}{rgb}{.71,0.21,0.01}
    \definecolor{citecolor}{rgb}{.12,.54,.11}

    % ANSI colors
    \definecolor{ansi-black}{HTML}{3E424D}
    \definecolor{ansi-black-intense}{HTML}{282C36}
    \definecolor{ansi-red}{HTML}{E75C58}
    \definecolor{ansi-red-intense}{HTML}{B22B31}
    \definecolor{ansi-green}{HTML}{00A250}
    \definecolor{ansi-green-intense}{HTML}{007427}
    \definecolor{ansi-yellow}{HTML}{DDB62B}
    \definecolor{ansi-yellow-intense}{HTML}{B27D12}
    \definecolor{ansi-blue}{HTML}{208FFB}
    \definecolor{ansi-blue-intense}{HTML}{0065CA}
    \definecolor{ansi-magenta}{HTML}{D160C4}
    \definecolor{ansi-magenta-intense}{HTML}{A03196}
    \definecolor{ansi-cyan}{HTML}{60C6C8}
    \definecolor{ansi-cyan-intense}{HTML}{258F8F}
    \definecolor{ansi-white}{HTML}{C5C1B4}
    \definecolor{ansi-white-intense}{HTML}{A1A6B2}

    % commands and environments needed by pandoc snippets
    % extracted from the output of `pandoc -s`
    \providecommand{\tightlist}{%
      \setlength{\itemsep}{0pt}\setlength{\parskip}{0pt}}
    \DefineVerbatimEnvironment{Highlighting}{Verbatim}{commandchars=\\\{\}}
    % Add ',fontsize=\small' for more characters per line
    \newenvironment{Shaded}{}{}
    \newcommand{\KeywordTok}[1]{\textcolor[rgb]{0.00,0.44,0.13}{\textbf{{#1}}}}
    \newcommand{\DataTypeTok}[1]{\textcolor[rgb]{0.56,0.13,0.00}{{#1}}}
    \newcommand{\DecValTok}[1]{\textcolor[rgb]{0.25,0.63,0.44}{{#1}}}
    \newcommand{\BaseNTok}[1]{\textcolor[rgb]{0.25,0.63,0.44}{{#1}}}
    \newcommand{\FloatTok}[1]{\textcolor[rgb]{0.25,0.63,0.44}{{#1}}}
    \newcommand{\CharTok}[1]{\textcolor[rgb]{0.25,0.44,0.63}{{#1}}}
    \newcommand{\StringTok}[1]{\textcolor[rgb]{0.25,0.44,0.63}{{#1}}}
    \newcommand{\CommentTok}[1]{\textcolor[rgb]{0.38,0.63,0.69}{\textit{{#1}}}}
    \newcommand{\OtherTok}[1]{\textcolor[rgb]{0.00,0.44,0.13}{{#1}}}
    \newcommand{\AlertTok}[1]{\textcolor[rgb]{1.00,0.00,0.00}{\textbf{{#1}}}}
    \newcommand{\FunctionTok}[1]{\textcolor[rgb]{0.02,0.16,0.49}{{#1}}}
    \newcommand{\RegionMarkerTok}[1]{{#1}}
    \newcommand{\ErrorTok}[1]{\textcolor[rgb]{1.00,0.00,0.00}{\textbf{{#1}}}}
    \newcommand{\NormalTok}[1]{{#1}}
    
    % Additional commands for more recent versions of Pandoc
    \newcommand{\ConstantTok}[1]{\textcolor[rgb]{0.53,0.00,0.00}{{#1}}}
    \newcommand{\SpecialCharTok}[1]{\textcolor[rgb]{0.25,0.44,0.63}{{#1}}}
    \newcommand{\VerbatimStringTok}[1]{\textcolor[rgb]{0.25,0.44,0.63}{{#1}}}
    \newcommand{\SpecialStringTok}[1]{\textcolor[rgb]{0.73,0.40,0.53}{{#1}}}
    \newcommand{\ImportTok}[1]{{#1}}
    \newcommand{\DocumentationTok}[1]{\textcolor[rgb]{0.73,0.13,0.13}{\textit{{#1}}}}
    \newcommand{\AnnotationTok}[1]{\textcolor[rgb]{0.38,0.63,0.69}{\textbf{\textit{{#1}}}}}
    \newcommand{\CommentVarTok}[1]{\textcolor[rgb]{0.38,0.63,0.69}{\textbf{\textit{{#1}}}}}
    \newcommand{\VariableTok}[1]{\textcolor[rgb]{0.10,0.09,0.49}{{#1}}}
    \newcommand{\ControlFlowTok}[1]{\textcolor[rgb]{0.00,0.44,0.13}{\textbf{{#1}}}}
    \newcommand{\OperatorTok}[1]{\textcolor[rgb]{0.40,0.40,0.40}{{#1}}}
    \newcommand{\BuiltInTok}[1]{{#1}}
    \newcommand{\ExtensionTok}[1]{{#1}}
    \newcommand{\PreprocessorTok}[1]{\textcolor[rgb]{0.74,0.48,0.00}{{#1}}}
    \newcommand{\AttributeTok}[1]{\textcolor[rgb]{0.49,0.56,0.16}{{#1}}}
    \newcommand{\InformationTok}[1]{\textcolor[rgb]{0.38,0.63,0.69}{\textbf{\textit{{#1}}}}}
    \newcommand{\WarningTok}[1]{\textcolor[rgb]{0.38,0.63,0.69}{\textbf{\textit{{#1}}}}}
    
    
    % Define a nice break command that doesn't care if a line doesn't already
    % exist.
    \def\br{\hspace*{\fill} \\* }
    % Math Jax compatability definitions
    \def\gt{>}
    \def\lt{<}
    % Document parameters
    \title{1.3\_jupyter-basics}
    
    
    

    % Pygments definitions
    
\makeatletter
\def\PY@reset{\let\PY@it=\relax \let\PY@bf=\relax%
    \let\PY@ul=\relax \let\PY@tc=\relax%
    \let\PY@bc=\relax \let\PY@ff=\relax}
\def\PY@tok#1{\csname PY@tok@#1\endcsname}
\def\PY@toks#1+{\ifx\relax#1\empty\else%
    \PY@tok{#1}\expandafter\PY@toks\fi}
\def\PY@do#1{\PY@bc{\PY@tc{\PY@ul{%
    \PY@it{\PY@bf{\PY@ff{#1}}}}}}}
\def\PY#1#2{\PY@reset\PY@toks#1+\relax+\PY@do{#2}}

\expandafter\def\csname PY@tok@w\endcsname{\def\PY@tc##1{\textcolor[rgb]{0.73,0.73,0.73}{##1}}}
\expandafter\def\csname PY@tok@c\endcsname{\let\PY@it=\textit\def\PY@tc##1{\textcolor[rgb]{0.25,0.50,0.50}{##1}}}
\expandafter\def\csname PY@tok@cp\endcsname{\def\PY@tc##1{\textcolor[rgb]{0.74,0.48,0.00}{##1}}}
\expandafter\def\csname PY@tok@k\endcsname{\let\PY@bf=\textbf\def\PY@tc##1{\textcolor[rgb]{0.00,0.50,0.00}{##1}}}
\expandafter\def\csname PY@tok@kp\endcsname{\def\PY@tc##1{\textcolor[rgb]{0.00,0.50,0.00}{##1}}}
\expandafter\def\csname PY@tok@kt\endcsname{\def\PY@tc##1{\textcolor[rgb]{0.69,0.00,0.25}{##1}}}
\expandafter\def\csname PY@tok@o\endcsname{\def\PY@tc##1{\textcolor[rgb]{0.40,0.40,0.40}{##1}}}
\expandafter\def\csname PY@tok@ow\endcsname{\let\PY@bf=\textbf\def\PY@tc##1{\textcolor[rgb]{0.67,0.13,1.00}{##1}}}
\expandafter\def\csname PY@tok@nb\endcsname{\def\PY@tc##1{\textcolor[rgb]{0.00,0.50,0.00}{##1}}}
\expandafter\def\csname PY@tok@nf\endcsname{\def\PY@tc##1{\textcolor[rgb]{0.00,0.00,1.00}{##1}}}
\expandafter\def\csname PY@tok@nc\endcsname{\let\PY@bf=\textbf\def\PY@tc##1{\textcolor[rgb]{0.00,0.00,1.00}{##1}}}
\expandafter\def\csname PY@tok@nn\endcsname{\let\PY@bf=\textbf\def\PY@tc##1{\textcolor[rgb]{0.00,0.00,1.00}{##1}}}
\expandafter\def\csname PY@tok@ne\endcsname{\let\PY@bf=\textbf\def\PY@tc##1{\textcolor[rgb]{0.82,0.25,0.23}{##1}}}
\expandafter\def\csname PY@tok@nv\endcsname{\def\PY@tc##1{\textcolor[rgb]{0.10,0.09,0.49}{##1}}}
\expandafter\def\csname PY@tok@no\endcsname{\def\PY@tc##1{\textcolor[rgb]{0.53,0.00,0.00}{##1}}}
\expandafter\def\csname PY@tok@nl\endcsname{\def\PY@tc##1{\textcolor[rgb]{0.63,0.63,0.00}{##1}}}
\expandafter\def\csname PY@tok@ni\endcsname{\let\PY@bf=\textbf\def\PY@tc##1{\textcolor[rgb]{0.60,0.60,0.60}{##1}}}
\expandafter\def\csname PY@tok@na\endcsname{\def\PY@tc##1{\textcolor[rgb]{0.49,0.56,0.16}{##1}}}
\expandafter\def\csname PY@tok@nt\endcsname{\let\PY@bf=\textbf\def\PY@tc##1{\textcolor[rgb]{0.00,0.50,0.00}{##1}}}
\expandafter\def\csname PY@tok@nd\endcsname{\def\PY@tc##1{\textcolor[rgb]{0.67,0.13,1.00}{##1}}}
\expandafter\def\csname PY@tok@s\endcsname{\def\PY@tc##1{\textcolor[rgb]{0.73,0.13,0.13}{##1}}}
\expandafter\def\csname PY@tok@sd\endcsname{\let\PY@it=\textit\def\PY@tc##1{\textcolor[rgb]{0.73,0.13,0.13}{##1}}}
\expandafter\def\csname PY@tok@si\endcsname{\let\PY@bf=\textbf\def\PY@tc##1{\textcolor[rgb]{0.73,0.40,0.53}{##1}}}
\expandafter\def\csname PY@tok@se\endcsname{\let\PY@bf=\textbf\def\PY@tc##1{\textcolor[rgb]{0.73,0.40,0.13}{##1}}}
\expandafter\def\csname PY@tok@sr\endcsname{\def\PY@tc##1{\textcolor[rgb]{0.73,0.40,0.53}{##1}}}
\expandafter\def\csname PY@tok@ss\endcsname{\def\PY@tc##1{\textcolor[rgb]{0.10,0.09,0.49}{##1}}}
\expandafter\def\csname PY@tok@sx\endcsname{\def\PY@tc##1{\textcolor[rgb]{0.00,0.50,0.00}{##1}}}
\expandafter\def\csname PY@tok@m\endcsname{\def\PY@tc##1{\textcolor[rgb]{0.40,0.40,0.40}{##1}}}
\expandafter\def\csname PY@tok@gh\endcsname{\let\PY@bf=\textbf\def\PY@tc##1{\textcolor[rgb]{0.00,0.00,0.50}{##1}}}
\expandafter\def\csname PY@tok@gu\endcsname{\let\PY@bf=\textbf\def\PY@tc##1{\textcolor[rgb]{0.50,0.00,0.50}{##1}}}
\expandafter\def\csname PY@tok@gd\endcsname{\def\PY@tc##1{\textcolor[rgb]{0.63,0.00,0.00}{##1}}}
\expandafter\def\csname PY@tok@gi\endcsname{\def\PY@tc##1{\textcolor[rgb]{0.00,0.63,0.00}{##1}}}
\expandafter\def\csname PY@tok@gr\endcsname{\def\PY@tc##1{\textcolor[rgb]{1.00,0.00,0.00}{##1}}}
\expandafter\def\csname PY@tok@ge\endcsname{\let\PY@it=\textit}
\expandafter\def\csname PY@tok@gs\endcsname{\let\PY@bf=\textbf}
\expandafter\def\csname PY@tok@gp\endcsname{\let\PY@bf=\textbf\def\PY@tc##1{\textcolor[rgb]{0.00,0.00,0.50}{##1}}}
\expandafter\def\csname PY@tok@go\endcsname{\def\PY@tc##1{\textcolor[rgb]{0.53,0.53,0.53}{##1}}}
\expandafter\def\csname PY@tok@gt\endcsname{\def\PY@tc##1{\textcolor[rgb]{0.00,0.27,0.87}{##1}}}
\expandafter\def\csname PY@tok@err\endcsname{\def\PY@bc##1{\setlength{\fboxsep}{0pt}\fcolorbox[rgb]{1.00,0.00,0.00}{1,1,1}{\strut ##1}}}
\expandafter\def\csname PY@tok@kc\endcsname{\let\PY@bf=\textbf\def\PY@tc##1{\textcolor[rgb]{0.00,0.50,0.00}{##1}}}
\expandafter\def\csname PY@tok@kd\endcsname{\let\PY@bf=\textbf\def\PY@tc##1{\textcolor[rgb]{0.00,0.50,0.00}{##1}}}
\expandafter\def\csname PY@tok@kn\endcsname{\let\PY@bf=\textbf\def\PY@tc##1{\textcolor[rgb]{0.00,0.50,0.00}{##1}}}
\expandafter\def\csname PY@tok@kr\endcsname{\let\PY@bf=\textbf\def\PY@tc##1{\textcolor[rgb]{0.00,0.50,0.00}{##1}}}
\expandafter\def\csname PY@tok@bp\endcsname{\def\PY@tc##1{\textcolor[rgb]{0.00,0.50,0.00}{##1}}}
\expandafter\def\csname PY@tok@fm\endcsname{\def\PY@tc##1{\textcolor[rgb]{0.00,0.00,1.00}{##1}}}
\expandafter\def\csname PY@tok@vc\endcsname{\def\PY@tc##1{\textcolor[rgb]{0.10,0.09,0.49}{##1}}}
\expandafter\def\csname PY@tok@vg\endcsname{\def\PY@tc##1{\textcolor[rgb]{0.10,0.09,0.49}{##1}}}
\expandafter\def\csname PY@tok@vi\endcsname{\def\PY@tc##1{\textcolor[rgb]{0.10,0.09,0.49}{##1}}}
\expandafter\def\csname PY@tok@vm\endcsname{\def\PY@tc##1{\textcolor[rgb]{0.10,0.09,0.49}{##1}}}
\expandafter\def\csname PY@tok@sa\endcsname{\def\PY@tc##1{\textcolor[rgb]{0.73,0.13,0.13}{##1}}}
\expandafter\def\csname PY@tok@sb\endcsname{\def\PY@tc##1{\textcolor[rgb]{0.73,0.13,0.13}{##1}}}
\expandafter\def\csname PY@tok@sc\endcsname{\def\PY@tc##1{\textcolor[rgb]{0.73,0.13,0.13}{##1}}}
\expandafter\def\csname PY@tok@dl\endcsname{\def\PY@tc##1{\textcolor[rgb]{0.73,0.13,0.13}{##1}}}
\expandafter\def\csname PY@tok@s2\endcsname{\def\PY@tc##1{\textcolor[rgb]{0.73,0.13,0.13}{##1}}}
\expandafter\def\csname PY@tok@sh\endcsname{\def\PY@tc##1{\textcolor[rgb]{0.73,0.13,0.13}{##1}}}
\expandafter\def\csname PY@tok@s1\endcsname{\def\PY@tc##1{\textcolor[rgb]{0.73,0.13,0.13}{##1}}}
\expandafter\def\csname PY@tok@mb\endcsname{\def\PY@tc##1{\textcolor[rgb]{0.40,0.40,0.40}{##1}}}
\expandafter\def\csname PY@tok@mf\endcsname{\def\PY@tc##1{\textcolor[rgb]{0.40,0.40,0.40}{##1}}}
\expandafter\def\csname PY@tok@mh\endcsname{\def\PY@tc##1{\textcolor[rgb]{0.40,0.40,0.40}{##1}}}
\expandafter\def\csname PY@tok@mi\endcsname{\def\PY@tc##1{\textcolor[rgb]{0.40,0.40,0.40}{##1}}}
\expandafter\def\csname PY@tok@il\endcsname{\def\PY@tc##1{\textcolor[rgb]{0.40,0.40,0.40}{##1}}}
\expandafter\def\csname PY@tok@mo\endcsname{\def\PY@tc##1{\textcolor[rgb]{0.40,0.40,0.40}{##1}}}
\expandafter\def\csname PY@tok@ch\endcsname{\let\PY@it=\textit\def\PY@tc##1{\textcolor[rgb]{0.25,0.50,0.50}{##1}}}
\expandafter\def\csname PY@tok@cm\endcsname{\let\PY@it=\textit\def\PY@tc##1{\textcolor[rgb]{0.25,0.50,0.50}{##1}}}
\expandafter\def\csname PY@tok@cpf\endcsname{\let\PY@it=\textit\def\PY@tc##1{\textcolor[rgb]{0.25,0.50,0.50}{##1}}}
\expandafter\def\csname PY@tok@c1\endcsname{\let\PY@it=\textit\def\PY@tc##1{\textcolor[rgb]{0.25,0.50,0.50}{##1}}}
\expandafter\def\csname PY@tok@cs\endcsname{\let\PY@it=\textit\def\PY@tc##1{\textcolor[rgb]{0.25,0.50,0.50}{##1}}}

\def\PYZbs{\char`\\}
\def\PYZus{\char`\_}
\def\PYZob{\char`\{}
\def\PYZcb{\char`\}}
\def\PYZca{\char`\^}
\def\PYZam{\char`\&}
\def\PYZlt{\char`\<}
\def\PYZgt{\char`\>}
\def\PYZsh{\char`\#}
\def\PYZpc{\char`\%}
\def\PYZdl{\char`\$}
\def\PYZhy{\char`\-}
\def\PYZsq{\char`\'}
\def\PYZdq{\char`\"}
\def\PYZti{\char`\~}
% for compatibility with earlier versions
\def\PYZat{@}
\def\PYZlb{[}
\def\PYZrb{]}
\makeatother


    % Exact colors from NB
    \definecolor{incolor}{rgb}{0.0, 0.0, 0.5}
    \definecolor{outcolor}{rgb}{0.545, 0.0, 0.0}



    
    % Prevent overflowing lines due to hard-to-break entities
    \sloppy 
    % Setup hyperref package
    \hypersetup{
      breaklinks=true,  % so long urls are correctly broken across lines
      colorlinks=true,
      urlcolor=urlcolor,
      linkcolor=linkcolor,
      citecolor=citecolor,
      }
    % Slightly bigger margins than the latex defaults
    
    \geometry{verbose,tmargin=1in,bmargin=1in,lmargin=1in,rmargin=1in}
    
    

    \begin{document}
    
    
    \maketitle
    
    

    
    \section{1.3 jupyter basics}\label{jupyter-basics}

This notebook will demonstrate the basics of working with jupyter
notebooks. We'll practice using the bash commands we learned in the
previous slide deck.

    \subsection{Creating and running
cells}\label{creating-and-running-cells}

First, create a new cell below this one by clicking on this cell and the
selecting "Insert"-\textgreater{}"Insert Cell Below". (NOTE: If you
accidentally double-click on this cell and go in \emph{edit mode}, you
can return to \emph{command mode} by clicking the "Run" button above.)

After you've created your new cell, type \texttt{2+3} and run the cell.
Play around with other arithmetic operations.

    \subsection{Running bash commands}\label{running-bash-commands}

\paragraph{\texorpdfstring{Practice some of your bash commands by adding
\texttt{\%\%bash} to the start of a
cell.}{Practice some of your bash commands by adding \%\%bash to the start of a cell.}}\label{practice-some-of-your-bash-commands-by-adding-bash-to-the-start-of-a-cell.}

For example, type these commands into a new cell and run it:

\begin{verbatim}
%%bash
pwd
\end{verbatim}

    \begin{Verbatim}[commandchars=\\\{\}]
{\color{incolor}In [{\color{incolor}1}]:} \PYZpc{}\PYZpc{}bash
        \PY{n+nb}{pwd}
\end{Verbatim}


    \begin{Verbatim}[commandchars=\\\{\}]
/Users/yinyiming/Applied-Bioinformatics/Module-1\_bash-jupyter-git

    \end{Verbatim}

    \paragraph{Examine the contents of the current
directory:}\label{examine-the-contents-of-the-current-directory}

\begin{verbatim}
%%bash
ls
\end{verbatim}

    \begin{Verbatim}[commandchars=\\\{\}]
{\color{incolor}In [{\color{incolor}2}]:} \PYZpc{}\PYZpc{}bash
        ls
\end{Verbatim}


    \begin{Verbatim}[commandchars=\\\{\}]
1.3\_jupyter-basics.ipynb
1.4\_working-with-files.ipynb
1.5\_redirection-and-pipes.ipynb
1.6\_awk.ipynb
data

    \end{Verbatim}

    \paragraph{\texorpdfstring{Examine the contents of the \texttt{data/art}
subdirectory}{Examine the contents of the data/art subdirectory}}\label{examine-the-contents-of-the-dataart-subdirectory}

\begin{verbatim}
%%bash
ls data/art
\end{verbatim}

    \begin{Verbatim}[commandchars=\\\{\}]
{\color{incolor}In [{\color{incolor}4}]:} \PYZpc{}\PYZpc{}bash
        ls data/art
\end{Verbatim}


    \begin{Verbatim}[commandchars=\\\{\}]
artists.txt

    \end{Verbatim}

    \paragraph{\texorpdfstring{Examine the context of the
\texttt{artists.txt}
file}{Examine the context of the artists.txt file}}\label{examine-the-context-of-the-artists.txt-file}

\begin{verbatim}
%%bash
cd data/art
less artists.txt
\end{verbatim}

    \begin{Verbatim}[commandchars=\\\{\}]
{\color{incolor}In [{\color{incolor}1}]:} \PYZpc{}\PYZpc{}bash
        \PY{n+nb}{cd} data/art
        less artists.txt
\end{Verbatim}


    \begin{Verbatim}[commandchars=\\\{\}]
KILIAN, Wolfgang	(1581-1662)	Baroque	German graphic artist (Augsburg)
KINSOEN, François-Joseph	(1771-1839)	Romanticism	Flemish painter
KISS, August Karl Edouard	(1802-1865)	Romanticism	German sculptor
KISS, Bálint	(1802-1868)	Romanticism	Hungarian painter (Pest)
KLEINER, Salomon	(1703-1761)	Baroque	German graphic artist
KLENZE, Leo von	(1784-1864)	Romanticism	German architect
KLESECKER, Justus (see GLESKER, Justus)	(c. 1615-1678)	Baroque	German sculptor (Franconia)
KLINGER, Max	(1857-1920)	Realism	German painter
KLOCKER, Hans	(active 1478-1500l)	Northern Renaissance	Austrian sculptor (South Tyrol)
KLODT, Mikhail Konstantinovich	(c. 1832-1902)	Realism	Russian painter (St. Petersburg)
KLODT, Pyotr Karlovich	(1805-1867)	Romanticism	Russian sculptor (St. Petersburg)
TACCA, Ferdinando	(1619-c. 1688)	Baroque	Italian sculptor (Florence)
TACCA, Pietro	(1577-1640)	Baroque	Italian sculptor (Florence)
TACCONE, Paolo (see PAOLO ROMANO)	(c. 1415-c. 1470)	Early Renaissance	Italian sculptor (Rome)
TADDEO DI BARTOLO	(1362/63-1422)	Medieval	Italian painter (Siena)
TADDEO DI FERRARA (see CRIVELLI, Taddeo)	(active 1451-c. 1479)	Early Renaissance	Italian illuminator (Ferrara)
TADOLINI, Scipione	(1822-1892)	Realism	Italian sculptor (Rome)
TALENTI, Francesco	(c. 1300-after 1369)	Medieval	Italian architect (Florence)
TALENTI, Simone	(c. 1330-after 1383)	Medieval	Italian architect (Florence)
TALPA, Bartolo	(active c. 1495)	Early Renaissance	Italian sculptor (Mantua)
TAMAGNI, Vincenzo	(1492-1530)	High Renaissance	Italian painter (San Gimignano)
TAMAGNINO, Antonio (see PORTA, Antonio della)	(active 1489–1519)	Early Renaissance	Italian sculptor
TAMM, Franz Werner von	(1658-1724)	Baroque	German painter
TANZIO DA VARALLO	(c. 1580-c. 1632)	Baroque	Italian painter
TARAVAL, Guillaume-Thomas-Raphaël	(1701-1750)	Baroque	French painter (Stockholm)
TARAVAL, Hugues	(1729-1785)	Rococo	French painter (Paris)
TARAVAL, Louis-Gustave	(1738-1794)	Neoclassicism	French graphic artist
TARBELL, Edmund Charles	(1862-1938)	Impressionism	American painter
TARCHIANI, Filippo	(1576-1645)	Baroque	Italian painter (Florence)
TARDIEU, Nicolas-Henry	(1674-1749)	Baroque	French graphic artist (Paris)
TARGONE, Cesare	(active 1575-1590)	Mannerism	Italian goldsmith
TARSIA, Antonio	(c. 1662-1739)	Baroque	Austrian sculptor (Venice)
TARUFFI, Emilio	(1633-1696)	Baroque	Italian painter (Bologna)
TASSAERT, Jean-Pierre-Antoine	(1727-1788)	Rococo	Flemish sculptor
TASSAERT, Octave	(1800-1874)	Romanticism	French painter
TASSEL, Jean	(1608-1667)	Baroque	French painter (Langres)
TASSI, Agostino	(1578-1644)	Baroque	Italian painter (Rome)
TAUNAY, Nicolas Antoine	(1755-1830)	Rococo	French painter
TEDESCO, Carlo (see SFERINI, Carlo Leopoldo)	(1652-1698)	Baroque	Italian painter (Verona)
TEDESCO, Giovanni Paolo (see SCHOR, Johann Paul)	(1615-1674)	Baroque	Italian painter (Rome)
TEERLINC, Levina	c. 1515-1576)	Mannerism	Flemish miniaturist (London)
TELEPY, Károly	(1828-1906)	Romanticism	Hungarian painter
TEMANZA, Tommaso	(1705-1789)	Rococo	Italian architect (Venice)
TEMPEL, Abraham van den	(1622/23-1672)	Baroque	Dutch painter (Amsterdam)
TEMPESTA, Antonio	(1555-1630)	Baroque	Italian painter (Florence)
TEMPESTA, Pietro (see MULIER, Pieter the Younger)	(1637-1701)	Baroque	Dutch painter (Italy)
TENERANI, Pietro	(1789-1869)	Neoclassicism	Italian sculptor
TENGNAGEL, Jan	(1584-1635)	Baroque	Dutch painter (Amsterdam)
TENIERS, Abraham	(1629-1670)	Baroque	Flemish painter (Antwerp)
TENIERS, David the Elder	(1582-1649)	Baroque	Flemish painter (Antwerp)
TENIERS, David the Younger	(1610-1690)	Baroque	Flemish painter (Antwerp)
TERBORCH, Gerard	(1617-1681)	Baroque	Dutch painter (Deventer)
TERBRUGGHEN, Hendrick	(1588-1629)	Baroque	Dutch painter (Utrecht)
TERILLI, Francesco	(active 1610-1630)	Baroque	Italian sculptor (Feltre)
TERRENI, Giuseppe Maria	(1739-1811)	Neoclassicism	Italian painter (Livorno)
TERZIO, Francesco	(c. 1523-1591)	Mannerism	Italian painter (Vienna)
TESSIN, Nicodemus the Elder	(1615-1681)	Baroque	Swedish architect (Stockholm)
TESSIN, Nicodemus the Younger	(1654-1728)	Baroque	Swedish architect (Stockholm)
TESTA, Pietro	(1611-1650)	Baroque	Italian painter (Rome)
TESTELIN, Henri	(1616-1695)	Baroque	French painter
TETRODE, Willem Danielsz van	(c. 1525-c. 1587)	Mannerism	Netherlandish sculptor
KNEBEL, Franz	(1809-1877)	Romanticism	Swiss painter (Rome)
KNELLER, Sir Godfrey	(1646-1723)	Baroque	English painter
KNIJFF, Jacob	(1639-1681)	Baroque	Dutch painter (London)
KNIJFF, Wouter	(c. 1607-after 1693)	Baroque	Dutch painter (Haarlem)
KNIP, Henriëtte	(1821-1909)	Realism	Dutch painter (Brussels)
KNIP, Henriëtte Geertruida	(1783-1842)	Romanticism	Dutch painter
KNIP, Josephus Augustus	(1777-1847)	Romanticism	Dutch painter
KNIP, Mattheus Derk	(1785-1845)	Romanticism	Dutch painter
KNIP, Nicolaas Frederik	(1741-1808)	Rococo	Dutch painter
KNOBELSDORFF, Georg Wenceslaus von	(1699-1753)	Baroque	German architect (Berlin)
KNOLLER, Martin	(1725-1804)	Neoclassicism	Austrian painter (Italy)
KNÜPFER, Nicolaus	(c. 1603-1655)	Baroque	Dutch painter
KNYFF, Jacob (see KNIJFF, Jacob)	(1639-1681)	Baroque	Dutch painter (London)
KNYFF, Wouter (see KNIJFF, Wouter)	(c. 1607-after 1693)	Baroque	Dutch painter (Haarlem)
KOBELL, Franz	(1749-1822)	Rococo	German painter (Munich)
KOBELL, Hendrik	(1751-1779)	Rococo	Dutch painter (Rotterdam)
KOBELL, Jan I	(1755-1833)	Rococo	Dutch graphic artist (Rotterdam)
KOBELL, Jan II	(1778-1814)	Neoclassicism	Dutch painter
KOBELL, Wilhelm von	(1766-1853)	Romanticism	German painter
KOBERGER, Anton	(c. 1440-1513)	Northern Renaissance	German graphic artist (Nuremberg)
KŘBKE, Christen	(1810-1848)	Romanticism	Danish painter
KOCH, Joseph Anton	(1768-1839)	Romanticism	Austrian painter
KOEBERGER, Wenzel (see COBERGHER, Wenceslas)	(c. 1560-1634)	Baroque	Flemish architect
KOEDIJCK, Isaack	(c. 1617-c. 1668)	Baroque	Dutch painter (Amsterdam)
KOEKKOEK, Barend Cornelis	(1803-1862)	Romanticism	Dutch painter
KOERBECKE, Johann	(c. 1420-1490)	Northern Renaissance	German painter (Münster)
KOETS, Roelof	(c. 1592-1655)	Baroque	Dutch painter (Haarlem)
KOKORINOV, Alexander Filippovich	(1726-1772)	Baroque	Russian architect (St. Petersburg)
KOLBE, Carl Wilhelm	(1759-1835)	Neoclassicism	German painter
KOLLONITSCH, Christian	(1730-1802)	Rococo	Austrian painter (Vienna)
KOLUNIĆ, Martin (see ROTA, Martino)	(c. 1520-1583)	Mannerism	Other painter (Vienna)
KOMPE, Jan ten (see COMPE, Jan ten)	(1713-1761)	Baroque	Dutch painter (Amsterdam)
KÖNIG, Johann	(1586-1642)	Baroque	German painter (Augsburg)
KONINCK, Daniël de	(1668-after 1720)	Baroque	Dutch painter (London)
KONINCK, David de (see CONINCK, David de)	(c. 1644-c. 1701)	Baroque	Flemish painter
KONINCK, Philips	(1619-1688)	Baroque	Dutch painter (Amsterdam)
KONINCK, Salomon	(1609-1656)	Baroque	Dutch painter (Amsterdam)
KONRAD von Soest	(active 1394-1422)	Northern Renaissance	German painter (Westphalie)
KOPISCH, August	(1799-1853)	Romanticism	German painter (Berlin)
AACHEN, Hans von	(1552-1615)	Mannerism	German painter
AAGAARD, Carl Frederik	(1833-1895)	Realism	Danish painter (Copenhagen)
ABADIA, Juan de la	(active 1470-1490)	Early Renaissance	Spanish painter (Huesca)
ABAQUESNE, Masséot	(c 1500-1564)	High Renaissance	French potter
ABBATE, Niccolò dell'	(1509-1571)	Mannerism	Italian painter (Bologna)
ABBATI, Giuseppe	(1836-1868)	Realism	Italian painter
ABBATINI, Guido Ubaldo	(c. 1602-1656)	Baroque	Italian painter (Rome)
ABBONDI, Antonio di Pietro (see SCARPAGNINO)	(c. 1465-1549)	High Renaissance	Italian architect (Venice)
ABEELE, Pieter van	(1608-1684)	Baroque	Dutch sculptor (Amsterdam)
ABILDGAARD, Nicolai	(1743-1809)	Romanticism	Danish painter
ABONDIO, Antonio	(1538-1591)	Mannerism	Italian sculptor (Vienna)
ACERO Y AREBO, Vicente	(c. 1677-1739)	Baroque	Spanish architect
ACHENBACH, Oswald	(1827-1905)	Realism	German painter (Dusseldorf)
ADAM, Albrecht	(1786-1862)	Romanticism	German painter (Munich)
ADAM, Lambert-Sigisbert	(1700-1759)	Baroque	French sculptor (Nancy)
ADAM, Nicolas-Sébastien	(1705-1778)	Baroque	French sculptor (Nancy)
ADAM, Robert	(1728-1792)	Baroque	Scottish architect (London)
ADAM-SALOMON, Antoine-Samuel	(1818-1881)	Realism	French sculptor (Paris)
ADELCRANTZ, Carl Fredrik	(1716-1796)	Neoclassicism	Swedish architect
ADEMOLLO, Luigi	(1764-1849)	Neoclassicism	Italian painter (Florence)
ADLER, Salomon	(1630-1709)	Baroque	German painter (Milan)
ADOLFZOON, Christoph (see ADOLPHI, Christoffel)	(c. 1631-1680)	Baroque	Dutch sculptor (Amsterdam)
ADOLPHI, Christoffel	(c. 1631-1680)	Baroque	Dutch sculptor (Amsterdam)
ADRIAEN VAN UTRECHT (see UTRECHT, Adriaen van)	(1599-1652)	Baroque	Flemish painter
ADRIAENSSEN, Alexander	(1587-1661)	Baroque	Flemish painter (Antwerp)
ADRIANO FIORENTINO	(ca. 1455-1499)	Early Renaissance	Italian sculptor (Florence)
AELST, Willem van	(1627-c. 1683)	Baroque	Dutch painter (Delft)
AENVANCK, Theodoor	(1633-1690)	Baroque	Flemish painter (Antwerp)
AERT VAN ORT	(active 1490-1536)	Northern Renaissance	Flemish glass painter
AERTSEN, Pieter	(1508-1575)	Northern Renaissance	Netherlandish painter (Amsterdam)
AGASSE, Jacques-Laurent	(1767-1849)	Neoclassicism	Swiss painter (London)
AGNOLO DA SIENA (see AGNOLO DI VENTURA)	(active 1311-1349)	Medieval	Italian sculptor (Siena)
AGNOLO DI POLO	(c. 1470-after 1498)	Early Renaissance	Italian sculptor (Florence)
AGNOLO DI VENTURA	(active 1311-1349)	Medieval	Italian sculptor (Siena)
AGOSTINO DI DUCCIO	(1418-1481)	Early Renaissance	Italian sculptor (Rimini)
AGOSTINO DI GIOVANNI	(active 1310-1347)	Medieval	Italian sculptor (Siena)
AGOSTINO VENEZIANO (see MUSI, Agostino dei)	(c. 1490-c. 1536)	High Renaissance	Italian graphic artist
AGRATE, Gian Francesco Ferrari d'	(1489-c. 1563)	High Renaissance	Italian sculptor (Parma)
AGUADO LÓPEZ, Antonio	(1764-1831)	Neoclassicism	Spanish architect (Madrid)
AIVAZOVSKY, Ivan Konstantinovich	(1817-1900)	Realism	Russian painter
AKEN, Joseph van	(c. 1699-1749)	Rococo	Flemish painter (London)
AKOTANTOS, Angelos	(?-c. 1457)	Medieval	Greek painter (Crete)
ALBA, Macrino d'	(c. 1460-c. 1528)	Early Renaissance	Italian painter (Alba)
ALBANI, Francesco	(1578-1660)	Baroque	Italian painter (Bologna)
ALBEREGNO, Jacobello	(died before1397)	Medieval	Italian painter (Venice)
ALBERTI, Antonio	(c. 1390-c. 1442)	Early Renaissance	Italian painter (Urbino)
ALBERTI, Cherubino	(1553-1615)	Mannerism	Italian painter (Rome)
ALBERTI, Giovanni	(1558-1601)	Mannerism	Italian painter (Rome)
ALBERTI, Leon Battista	(1404-1472)	Early Renaissance	Italian architect
ALBERTINELLI, Mariotto	(1474-1515)	Early Renaissance	Italian painter (Florence)

    \end{Verbatim}

    \paragraph{What year was the painter Giuseppe Abbati born
in?}\label{what-year-was-the-painter-giuseppe-abbati-born-in}

Find and correct the error...

\begin{verbatim}
%%bash
cd data/art
grep Abbati artists.txt
\end{verbatim}

    \begin{Verbatim}[commandchars=\\\{\}]
{\color{incolor}In [{\color{incolor}2}]:} \PYZpc{}\PYZpc{}bash
        \PY{n+nb}{cd} data/art
        grep \PYZhy{}i Abbati artists.txt
\end{Verbatim}


    \begin{Verbatim}[commandchars=\\\{\}]
ABBATI, Giuseppe	(1836-1868)	Realism	Italian painter
ABBATINI, Guido Ubaldo	(c. 1602-1656)	Baroque	Italian painter (Rome)

    \end{Verbatim}

    \paragraph{\texorpdfstring{How many Italian painters are listed in
\texttt{artists.txt}?}{How many Italian painters are listed in artists.txt?}}\label{how-many-italian-painters-are-listed-in-artists.txt}

Find and correct the error...

\begin{verbatim}
%%bash
cd /data/art
grep -c Italian painters artists.txt
\end{verbatim}

    \begin{Verbatim}[commandchars=\\\{\}]
{\color{incolor}In [{\color{incolor}3}]:} \PYZpc{}\PYZpc{}bash
        \PY{n+nb}{cd} data/art
        grep \PYZhy{}c \PY{l+s+s1}{\PYZsq{}Italian painter\PYZsq{}} artists.txt
\end{Verbatim}


    \begin{Verbatim}[commandchars=\\\{\}]
23

    \end{Verbatim}

    \subsection{Homework}\label{homework}

These problems will test your understanding of both basic bash commands
and of the jupyter notebook. Answer each of the questions below by
creating and executing a new code cell underneath each question.

    \subsubsection{\texorpdfstring{HW1: Use the \texttt{ls} command to sort
all files in \texttt{data/art} by file
size.}{HW1: Use the ls command to sort all files in data/art by file size.}}\label{hw1-use-the-ls-command-to-sort-all-files-in-dataart-by-file-size.}

    \begin{Verbatim}[commandchars=\\\{\}]
{\color{incolor}In [{\color{incolor}4}]:} \PYZpc{}\PYZpc{}bash
        ls data/art
\end{Verbatim}


    \begin{Verbatim}[commandchars=\\\{\}]
artists.txt

    \end{Verbatim}

    \subsubsection{\texorpdfstring{HW2: Count the number of artists in
\texttt{artists.txt} \emph{excluding} French
painters.}{HW2: Count the number of artists in artists.txt excluding French painters.}}\label{hw2-count-the-number-of-artists-in-artists.txt-excluding-french-painters.}

    \begin{Verbatim}[commandchars=\\\{\}]
{\color{incolor}In [{\color{incolor}7}]:} \PYZpc{}\PYZpc{}bash
        \PY{n+nb}{cd} data/art
        grep \PYZhy{}v \PY{l+s+s1}{\PYZsq{}French painter\PYZsq{}} \PYZhy{}c artists.txt
\end{Verbatim}


    \begin{Verbatim}[commandchars=\\\{\}]
144

    \end{Verbatim}

    \subsubsection{\texorpdfstring{HW3: List only the csv files in
\texttt{data/tmp}.}{HW3: List only the csv files in data/tmp.}}\label{hw3-list-only-the-csv-files-in-datatmp.}

    \begin{Verbatim}[commandchars=\\\{\}]
{\color{incolor}In [{\color{incolor}8}]:} \PYZpc{}\PYZpc{}bash
        \PY{n+nb}{cd} data/tmp
        ls *.csv
\end{Verbatim}


    \begin{Verbatim}[commandchars=\\\{\}]
file192.csv
file33.csv
file96.csv

    \end{Verbatim}


    % Add a bibliography block to the postdoc
    
    
    
    \end{document}
